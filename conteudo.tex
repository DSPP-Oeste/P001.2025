\section{Enquadramento}\label{sec:enquadramento}

A atividade do \gls{dspp} da \gls{ulso} requer uma abordagem sistematizada e documentada para a definição e implementação de procedimentos técnico-operacionais. Estes procedimentos visam assegurar coerência, qualidade e rastreabilidade nas intervenções em saúde pública, bem como promover a equidade e eficiência na atuação dos profissionais do \gls{dspp}.

O presente documento estabelece as normas e orientações para a proposição, elaboração, aprovação, alteração e revogação de procedimentos internos do \gls{dspp}. Este “procedimento de procedimentos” visa garantir que todos os documentos produzidos sigam uma metodologia uniforme e criteriosamente validada, promovendo a melhoria contínua dos processos organizacionais e técnicos no âmbito da saúde pública.

Este procedimento é aplicável a todas as tipologias de procedimentos operacionais internos, incluindo os de carácter clínico, técnico, administrativo ou transversal, independentemente da área temática ou programa de saúde a que digam respeito.


\section{Objetivo}\label{sec:objetivo}

O presente procedimento tem como objetivo normatizar o processo de criação, revisão, alteração e revogação de procedimentos internos no âmbito do \gls{dspp}


\section{Âmbito}\label{sec:ambito}

Entende-se por procedimento interno qualquer documento normativo, aprovado e homologado, que estabeleça de forma sistemática as etapas, responsabilidades, critérios técnicos e operacionais a observar na execução de uma determinada atividade no âmbito do \gls{dspp}.

Este procedimento aplica-se a todos os procedimentos internos desenvolvidos no seio do \gls{dspp}, independentemente da sua natureza (clínica, técnica, administrativa ou transversal) ou da área temática a que respeitem. Abrange igualmente os procedimentos produzidos no contexto dos programas de saúde sob coordenação do \gls{dspp}.

Estão incluídos no âmbito deste procedimento:

\begin{itemize}
    \item a criação de novos procedimentos por proposta de profissionais ou órgãos do \gls{dspp};
    \item a alteração ou atualização de procedimentos existentes;
    \item a revogação formal de procedimentos desatualizados, obsoletos ou redundantes;
    \item a definição da estrutura e conteúdo obrigatório de cada procedimento;
    \item os critérios de validação técnica e homologação hierárquica.
\end{itemize}


\section{Destinatários}\label{sec:destinatarios}

O presente procedimento destina-se a todos os profissionais do \gls{dspp}, incluindo:

\begin{itemize}
    \item Direção do Departamento;
    \item membros do Conselho Técnico;
    \item coordenadores e elementos das equipas de coordenação dos programas de saúde sob responsabilidade do \gls{dspp};
    \item médicos, enfermeiros, técnicos superiores de saúde, assistentes técnicos e outros profissionais integrados nos serviços e unidades funcionais do departamento;
    \item colaboradores externos que, mediante delegação ou participação formal, estejam envolvidos na elaboração, revisão ou implementação de procedimentos.
\end{itemize}

Este procedimento é de cumprimento obrigatório por todos os profissionais do \gls{dspp} envolvidos na elaboração, revisão ou implementação de procedimentos internos, bem como pelas equipas de coordenação dos programas operacionais da \gls{ulso} que atuem sob responsabilidade direta ou indireta do departamento.


\section{Definição}\label{sec:definicao}

    \subsection{Definição Funcional}\label{subsec:definicao-funcional}

    O presente procedimento define um instrumento que:
    \begin{itemize}
        \item define as etapas necessárias à criação, revisão, alteração e revogação de procedimentos internos;
        \item estrutura os fluxos de validação técnica e decisão hierárquica entre o Conselho Técnico e a Direção do \gls{dspp};
        \item assegura a uniformidade e consistência dos documentos produzidos, promovendo a sua legibilidade e aplicabilidade prática;
        \item garante a rastreabilidade das decisões e intervenções associadas a cada procedimento.
    \end{itemize}
    
    \subsection{Limites de Entrada no Processo}\label{subsec:limites-de-entrada-no-processo}

    Este procedimento tem início com a submissão formal da proposta de criação, alteração ou revogação de um procedimento ao Conselho Técnico, por correio eletrónico. Esta proposta deve identificar de forma clara a necessidade sentida, o objetivo do procedimento e os seus limites previstos.

    \subsection{Limites de Saída do Processo}\label{subsec:limites-de-saida-do-processo}

    O procedimento termina com a homologação pelo Diretor do Departamento e consequente publicação do procedimento final na pasta partilhada oficial do \gls{dspp}, acompanhada de notificação por e-mail aos profissionais do departamento.

    \subsection{Limites Marginais}\label{subsec:limites-marginais}

    \begin{itemize}
        \item Registo da atividade relacionada com a elaboração do procedimentos (e dos procedimentos em si)
        \item Avaliação e Monitorização dos procedimentos
    \end{itemize}


\section{Responsabilidades}\label{sec:responsabilidades}

\begin{itemize}
    \item \textbf{Profissionais do \gls{dspp}}: Podem propor a criação, alteração ou revogação de procedimentos, devendo submeter proposta formal ao Conselho Técnico, com identificação clara da necessidade, objetivo e âmbito do procedimento.
    \item \textbf{Equipas de Coordenação dos Programas}: Podem propor e colaborar na elaboração de procedimentos no âmbito do respetivo programa. Assumem a redação técnica sempre que designadas para esse efeito pelo Conselho Técnico.
    \item \textbf{Conselho Técnico}: Recebe, analisa e delibera sobre as propostas de procedimento. Designa a equipa responsável pela sua elaboração, define prazos e acompanha o desenvolvimento do documento. Revê tecnicamente o procedimento e propõe alterações, submetendo a versão final ao Diretor do \gls{dspp} para homologação. Em caso de revogação, emite parecer e assegura a remoção do procedimento do repositório oficial.
    \item \textbf{Diretor do \gls{dspp}}: Homologa os procedimentos após apreciação da versão final. Pode solicitar alterações e devolvê-las ao Conselho Técnico. Tem competência para aprovar a revogação de procedimentos e determinar a sua entrada em vigor.
\end{itemize}


\section{Descrição do procedimento}\label{sec:descricao-do-procedimento}

\subsection{Criação de um novo procedimento}\label{sec:criacao-do-procedimento}

\begin{enumerate}
  \item Qualquer profissional, equipa de coordenação de programa, o \textbf{Conselho Técnico} ou o \textbf{Diretor do \gls{dspp}} pode propor a criação de um novo procedimento.
  \item A proposta deve ser enviada por e-mail ao \textbf{Conselho Técnico}, com a identificação da necessidade sentida, o objetivo do procedimento e os seus limites.
  \item Após receção da proposta, o \textbf{Conselho Técnico} avalia a proposta e decide sobre a sua pertinência;
  \item Em caso de indeferimento, o \textbf{Conselho Técnico} comunica a decisão ao propositor, dando conhecimento ao \textbf{Diretor do \gls{dspp}};
  \item Em caso de deferimento, o \textbf{Conselho Técnico}:
  \begin{enumerate}
    \item Designa uma \textbf{Equipa Redatora}, responsável pela elaboração do procedimento;
    \item Designa um \textbf{Revisor}, responsável pela revisão do procedimento, que pode ser o próprio conselho técnico ou um elemento externo a este;
    \item Define um prazo para entrega do primeiro draft do documento e um prazo para finalização dos trabalhos;
    \item Atribui um título ao Procedimento;
    \item Atribui um número ao Procedimento, no formato PXXX.AAAA, em que XXX é um número sequencial composto por 3 digitos e AAAA correspondem aos 4 dígitos do ano corrente;
    \item Regista o procedimento na \textit{tabela dos procedimentos}, disponível na \href{https://snspt-my.sharepoint.com/:l:/r/personal/93839_ulso_min-saude_pt/Lists/Lista%20de%20Procedimentos?e=zkbTNC}{Lista de Procedimentos}, preenchendo as colunas de forma apropriada;
    \item Comunica, via email, a decisão ao propositor e à equipa redatora, dando conhecimento ao \textbf{Diretor do \gls{dspp}};
  \end{enumerate}
  \item A \textbf{Equipa Redatora} elabora o procedimento, seguindo o modelo oficial e cujas instruções se encontram no \refanexo{anexo-estrutura-modelo-procedimento}
  \item O documento (sem formatação final) é enviado por e-mail ao \textbf{Conselho Técnico}.
  \item O \textbf{Conselho Técnico} analisa e propõe alterações, se aplicável. A comunicação deve seguir sempre na mesma cadeia de e-mails.
  \item Após validação, o \textbf{Conselho Técnico} formata o documento e submete-o ao \textbf{Diretor do \gls{dspp}} para homologação.
  \item O Diretor pode sugerir alterações e devolvê-las ao \textbf{Conselho Técnico}.
  \item O processo de revisão pode ser repetido até aprovação final.
  \item Após homologação do documento pelo \textbf{Diretor do \gls{dspp}}, o \textbf{Conselho Técnico}:
  \begin{enumerate}
    \item Atualiza o ficheiro "metadata.tex" com as seguintes informações: Versão 1.0.0; Data do procedimento com a data em que o documento foi homologado pelo Diretor do \gls{dspp}; Tabela de controlo de versões com o revisor, quem homologou o documento e a data de homologação.
    \item publica o procedimento na pasta partilhada do \gls{dspp}, na diretoria "/Procedimentos e Legislação";
    \item envia, via email, notificação da existência de um novo procedimento a todos os profissionais.
  \end{enumerate}
  
   e comunica-o, via e-mail, a todos os profissionais.
\end{enumerate}

\subsection{Alteração de um procedimento existente}

\begin{enumerate}
  \item Qualquer profissional do \gls{dspp} pode propor alterações a um procedimento existente.
  \item A proposta deve ser enviada ao \textbf{Conselho Técnico}, por e-mail, com:
  \begin{enumerate}
    \item Justificação fundamentada para a alteração (racional da proposta);
    \item Documento original em anexo, com as alterações devidamente assinaladas (em comentário ou controlo de alterações);
  \end{enumerate}
  \item Após receção da proposta, o \textbf{Conselho Técnico} avalia a pertinência da mesma.
  \item Em caso de indeferimento, comunica a decisão ao propositor, dando conhecimento ao Diretor do \gls{dspp}.
  \item Em caso de deferimento, o \textbf{Conselho Técnico}:
  \begin{enumerate}
    \item Decide se, após validação, procede ele próprio à redação da nova versão ou se designa uma nova \textbf{Equipa Redatora};
    \item Define um prazo para entrega da versão revista;
    \item Comunica, via email, a decisão ao propositor e, se aplicável, à nova \textbf{Equipa Redatora}, dando conhecimento ao \textbf{Diretor do \gls{dspp}};
    \item Verte as alterações no "changelog.md" e atribui um novo número de versão, que segue as regras explanadas no \refanexo{anexo-regras-versionamento}
    \item Envia a nova versão para o \textbf{Diretor do \gls{dspp}} para homologação.
  \end{enumerate}
  \item Após a homologação do documento pelo \textbf{Diretor do \gls{dspp}}, o \textbf{Conselho Técnico} publica a nova versão na pasta partilhada do \gls{dspp}.
\end{enumerate}


\subsection*{Revogação de um procedimento}

\begin{enumerate}
  \item A revogação de um procedimento pode ser proposta pelo \textbf{Conselho Técnico} ou pelo Diretor do \gls{dspp}.
  \item Quando a proposta parte do \textbf{Diretor do \gls{dspp}}, este comunica formalmente, por e-mail, a intenção de revogação ao \textbf{Conselho Técnico}.
  \item O \textbf{Conselho Técnico} avalia a proposta e emite parecer fundamentado.
  \item Quando a proposta parte do \textbf{Conselho Técnico}, este procede diretamente à emissão do parecer.
  \item O parecer é submetido ao Diretor do \gls{dspp}.
  \item O Diretor pode promulgar a revogação, solicitar esclarecimentos ou reverter a decisão de revogação.
  \item Uma vez promulgada a revogação, o \textbf{Conselho Técnico}:
  \begin{enumerate}
    \item Atualiza a Lista de Procedimentos
    \item Retira o procedimento revogado da pasta partilhada do \gls{dspp};
    \item Envia email a todos os profissionais do \gls{dspp}, informando da revogação do procedimento.
  \end{enumerate}
\end{enumerate}



\section{Situações de Excepção}\label{sec:situacoes-de-excepcao}

Este procedimento não se aplica nos seguintes contextos:

\begin{itemize}
  \item Procedimentos cuja elaboração, validação ou publicação se encontrem reguladas por entidades externas ao \gls{dspp}, como Ordem dos Médicos, Direção-Geral da Saúde, Departamento de Saúde Pública Regional ou \gls{ulso} ou outras estruturas superiores;
  \item Normas técnicas ou instruções operacionais impostas por obrigação legal ou contratual, cuja adoção pelo \gls{dspp} seja obrigatória e imediata;
  \item Situações urgentes em que a Direção do \gls{dspp}, por decisão fundamentada, determine um circuito simplificado ou excecional de aprovação de um procedimento.
\end{itemize}

Nestas situações, o \textbf{Conselho Técnico} deve ser informado e pode emitir parecer retrospetivo, se aplicável.


\section{Monitorização e Avaliação}\label{sec:monitorizacao-e-avaliacao}

\subsection*{6. Monitorização e Avaliação}

A eficácia e adequação do presente procedimento serão monitorizadas de forma contínua pelo \textbf{Conselho Técnico}, enquanto entidade responsável pela gestão e supervisão dos procedimentos internos do \gls{dspp}.

\begin{itemize}
  \item A revisão formal deste procedimento deverá ocorrer, no máximo, a cada 3 anos, ou sempre que se verifiquem alterações significativas na organização interna, no modelo de procedimentos, ou nas necessidades do \gls{dspp}.
  
  \item O processo de monitorização será orientado para a melhoria contínua e baseado nos seguintes critérios:
  \begin{itemize}
    \item Número de procedimentos criados, revistos ou revogados anualmente;
    \item Cumprimento dos prazos definidos para entrega e homologação dos procedimentos;
    \item Conformidade dos procedimentos com o modelo oficial, incluindo a correta estruturação e preenchimento de todas as secções;
    \item Número de pedidos de correção ou reformulação durante a fase de validação;
    \item Existência de procedimentos desatualizados ou com mais de 3 anos sem revisão.
  \end{itemize}

  \item Para efeitos de registo e controlo, o \textbf{Conselho Técnico} deverá manter atualizada a \textit{Lista dos Procedimentos} com data da última revisão, versão em vigor e responsáveis envolvidos.

  \item Sempre que se identifiquem fragilidades ou oportunidades de melhoria relevantes no funcionamento deste procedimento, o \textbf{Conselho Técnico} poderá propor a sua alteração, iniciando para tal o circuito habitual de revisão de procedimentos.

  \item A avaliação anual será, sempre que possível, integrada no relatório de atividades do \gls{dspp}, com uma secção específica relativa à gestão dos procedimentos internos.
\end{itemize}

