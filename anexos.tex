\anexo{Estrutura e Modelo de Procedimento}\label{anexo-estrutura-modelo-procedimento}
\setlist{topsep=1em, itemsep=1em}

Os procedimentos internos do \gls{dspp} devem seguir o modelo oficial disponibilizado pelo \textbf{Conselho Técnico}, que inclui os seguintes campos/títulos estruturados:

\begin{itemize}
  \item \textbf{Enquadramento}: contextualiza o procedimento no âmbito da necessidade sentida e da missão e objetivos do \gls{dspp}, podendo incluir referências normativas, estratégicas, legislativas ou operacionais.
  
  \item \textbf{Objetivo}: define, de forma concisa e clara, o que se pretende alcançar com o procedimento.

  \item \textbf{Âmbito}: delimita o espaço de aplicação do procedimento, indicando as áreas, serviços ou situações em que se aplica.

  \item \textbf{Destinatários}: identifica os profissionais, grupos ou entidades a quem se destina o procedimento.

  \item \textbf{Definição}: apresenta os conceitos essenciais ou definições relevantes para a compreensão e correta aplicação do procedimento.

  \item \textbf{Definição Funcional}: descreve o processo regulado, as suas principais funções e o seu posicionamento operacional dentro do \gls{dspp}.

  \item \textbf{Limites de Entrada no Processo}: define o momento e a condição concreta em que se inicia a aplicação do procedimento.

  \item \textbf{Limites de Saída do Processo}: define o ponto terminal do procedimento, ou seja, quando se considera concluído.

  \item \textbf{Limites Marginais}: identifica elementos adjacentes ao procedimento, com os quais interage mas que não estão diretamente sob o seu controlo.

  \item \textbf{Responsabilidades}: lista os atores envolvidos e as suas funções específicas em cada etapa do procedimento.

  \item \textbf{Descrição do Procedimento}: apresenta as etapas sequenciais do processo regulado, em formato textual. Pode conter  gráficos (ex: fluxograma) que deverão constar em anexo.

  \item \textbf{Situações de Exceção}: identifica exceções à regra geral definida no procedimento, e os respetivos critérios de atuação.

  \item \textbf{Monitorização e Avaliação}: define os indicadores, métodos e frequência de monitorização da aplicação do procedimento.

  \item \textbf{Registo de Alterações}: apresenta o histórico de versões do procedimento, com data, responsáveis e principais alterações efetuadas.

  \item \textbf{Anexos}: inclui documentos complementares ou auxiliares à aplicação do procedimento, como formulários, tabelas, normas, quadros de apoio e fluxogramas.
\end{itemize}

\pagebreak

\anexo{Regras para o versionamento}\label{anexo-regras-versionamento}

\subsection*{Objetivo}

Este anexo estabelece as regras para a atribuição e atualização dos números de versão dos procedimentos internos do \gls{dspp}, garantindo coerência, rastreabilidade e clareza sobre o tipo de alterações realizadas em cada versão.

\subsection*{Formato da versão}

O número de versão segue o padrão \textbf{X.Y.Z}, onde:

\begin{itemize}
  \item \textbf{X} — Versão principal (\textit{major}): alterações estruturais ou conceptuais que tornam a nova versão incompatível com a anterior.
  \item \textbf{Y} — Versão secundária (\textit{minor}): adições ou modificações relevantes, mas retrocompatíveis com a versão anterior.
  \item \textbf{Z} — Versão de correção (\textit{patch}): alterações editoriais, correções de erros ou ajustes menores que não modificam o conteúdo funcional do procedimento.
\end{itemize}

\subsection*{Regras de incrementação}

\begin{itemize}
  \item A versão é iniciada como \texttt{1.0.0} aquando da publicação oficial do procedimento.
  \item Incrementa-se \texttt{Z} (ex: \texttt{1.0.1}) quando são feitas apenas:
  \begin{itemize}
    \item correções ortográficas ou gramaticais;
    \item ajustes de formatação;
    \item atualização de hiperligações ou contactos.
  \end{itemize}
  \item Incrementa-se \texttt{Y} (ex: \texttt{1.1.0}) quando são:
  \begin{itemize}
    \item adicionadas novas secções ou campos ao procedimento;
    \item feitas alterações ao conteúdo que mantêm a compatibilidade com a versão anterior;
    \item atualizados fluxos ou responsabilidades sem impactar a estrutura global.
  \end{itemize}
  \item Incrementa-se \texttt{X} (ex: \texttt{2.0.0}) quando:
  \begin{itemize}
    \item há reestruturação profunda do procedimento;
    \item são alteradas premissas, objetivos ou âmbitos;
    \item o procedimento se torna incompatível com versões anteriores (ex: eliminação de etapas críticas, alteração do processo base).
  \end{itemize}
\end{itemize}

\subsection*{Registo das versões}

Todas as versões devem ser registadas na secção “\textbf{Registo de Alterações}” do procedimento, com indicação da data, autores/revisores e descrição resumida das alterações.

%\pagebreak
%\anexo{Fluxograma do processo de criação de novo procedimento}\label{anexo-fluxograma-novo-procedimento}

