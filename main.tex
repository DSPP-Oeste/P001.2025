%% -----------------------------------------------------------------------------
%% Copyright (c) 2025 por Estêvão Soares dos Santos,
%% em colaboração com o Departamento de Saúde Pública e das Populações do Oeste (DSPPO)
%%
%% Este ficheiro LaTeX constitui código-fonte protegido por direitos de autor.
%% É licenciado sob os termos da Creative Commons CC BY-NC-SA 4.0.
%% O seu uso, modificação e redistribuição são permitidos exclusivamente para fins não comerciais,
%% desde que seja atribuída a autoria e quaisquer obras derivadas sejam partilhadas sob a mesma licença.
%%
%% É expressamente proibida a remoção desta nota de atribuição,
%% nos termos da licença CC BY-NC-SA 4.0.
%%
%% NOTA: Todos os logótipos, emblemas, brasões e imagens incluídos neste documento
%% são propriedade da República Portuguêsa e estão protegidos por legislação própria.
%% O uso desses elementos fora do contexto legal aplicável não é autorizado por este documento.
%%
%% SPDX-License-Identifier: CC-BY-NC-SA-4.0
%% -----------------------------------------------------------------------------

%% -----------------------------------------------------------------------------
%% COMO UTILIZAR:
%%  - main.tex:        Este ficheiro contem o template do documento. Genéricamente não deve ser alterado.
%%  - metadata.tex:    Metadados do documento (titulo, data, autor, ficha tecnica, etc...)
%%  - changelog.md:    Contem o registo das alterações realizadas no documento. Está em formato markdown
%%                     e segue a norma Conventional Changelog
%%  - referencias.bib: Ficheiro com as referencias bibliográficas
%%  - conteudo.tex:    Ficheiro onde deve ser "escrito" o documento per se
%%  - siglas.tex:      Ficheiro com as siglas e acrónimos que serão usados no documento.
%%  - media/*          Diretoria com as imagens usadas no template. Outras imagens usadas no documento
%%                     devem ser colocadas aqui.
%%  - LICENSE.txt:     Ficheiro com a licença do template. Não deve ser apagado
%% ------------------------------------------------------------------------------


\documentclass[a4paper,11pt]{report}

% Language setting
\usepackage[utf8]{inputenc}
\usepackage[T1]{fontenc}
\usepackage[portuguese]{babel}
\usepackage{csquotes}
\usepackage{markdown}
\usepackage{longtable}
\usepackage{setspace}


% Set page size and margins
\usepackage[a4paper,top=3.25cm,bottom=2.54cm,left=1.5cm,right=1.5cm,marginparwidth=1.25cm, headheight=1.5cm]{geometry}

% Useful packages
\usepackage[backend=biber, style=numeric]{biblatex}
\addbibresource{referencias.bib}
\usepackage{amsmath}
\usepackage{graphicx} % Required for inserting images
\usepackage[colorlinks=true, allcolors=black,plainpages=false,pdfpagelabels]{hyperref}
\usepackage{multicol}
\usepackage{tikz}
\usepackage{soul}
\usepackage{enumitem}
\usepackage[table]{xcolor}
\usepackage[numbered,open,openlevel=1]{bookmark}


% Font setup - Using a lightweight sans-serif
% Option 1: Source Sans Pro (very similar to Calibri)
\usepackage[light]{sourcesanspro}
\renewcommand{\familydefault}{\sfdefault}

% Option 2: Uncomment these for Tex Gyre Heros (Helvetica-like font)
%\usepackage{tgheros}
%\renewcommand{\familydefault}{\sfdefault}

% Option 3: Uncomment these for Fira Sans Light
%\usepackage[light]{FiraSans}
%\renewcommand{\familydefault}{\sfdefault}

% option 4: Uncomment these for helvetica clone
%\usepackage[light]{helvet}
%\renewcommand{\familydefault}{\sfdefault}

% siglas e acronimos package
\usepackage[acronym]{glossaries}

\newglossarystyle{long3colbold}{
  \setglossarystyle{long3col} % herda estrutura do estilo original

  % Modifica só o conteúdo das células
  \renewcommand*{\glossentry}[2]{%
    \textbf{\glsentryshort{##1}} & \glsentrydesc{##1} & ##2\\
  }

  \renewcommand*{\subglossentry}[3]{%
    \textbf{\glsentryshort{##2}} & \glsentrydesc{##2} & ##3\\
  }

  % Remove o ambiente center
  \renewenvironment{theglossary}
    {\begin{longtable}{@{}p{2cm}p{12.5cm}p{1.5cm}@{}}}
    {\end{longtable}}
}
\setglossarystyle{long3colbold}
\makeglossaries


% Load the metadata
%% ---------------------------------------------------------------------------
%% METADADOS DO DOCUMENTO
%% ---------------------------------------------------------------------------
%% 
%% Este ficheiro deve ser alterado/preenchido conforme os dados do documento
%%


% Metadata for Procedimento
\def\procedimentoTitulo{Procedimento para a proposição, elaboração, aprovação, alteração e revogação de procedimentos internos do DSPP}
\def\procedimentoNumero{P001.2025}
\def\procedimentoVersao{1.0.0}
\def\procedimentoData{03/03/2025}
\def\procedimentoDestinatarios{Todos os Profissionais do Departamento de Saúde Pública e das Populações}
\def\procedimentoPalavrasChave{Procedimentos}

% estado do documento - comentar a linha para tirar marca de agua DRAFT
%\def\estadodraft{}%

% Authors defined as a list
\def\listarAutores{%
  \begin{itemize}[label={}, leftmargin=0pt]
    \item Estêvão Soares dos Santos (coordenador)
    \item Leopoldina Moreira (autor)
    \item Sandra Silva (autor)
    \item Susana Alves (autor)
  \end{itemize}
}

% Version control as a table
\def\tabelaControloVersao{%
    \begin{tabular}{|p{3cm}|p{3cm}|p{3cm}|p{3cm}|}
        \hline
        \rowcolor{tableheader} 
        \textbf{Versão} & \textbf{Revisor} & \textbf{Aprovador} & \textbf{Data} \\
    \hline
    1.0.0 & Tiago Carvalho & Nuno Rodrigues & 17/06/2025 \\
    \hline
    \end{tabular}
}

% Componentes do template - Comentar a linha caso não existam anexos
\def\usaranexos{} % ANEXOS
\def\usarsiglas{} % Siglas e acronimos
\def\usarindicefiguras{}
\def\usarindicequadros{}


% Load siglas.tex
%% ----------------------------------------------------------------------------
% COMANDOS:
%
%    \newacronym{ulso}{ULSO}{Unidade Local de Saúde do Oeste}
%      Parametro 1. O label para referir no texto com o comando gls{}. Neste exemplo, gls{ulso}
%      Parametro 2. A sigla/acrónimo. Geralmente em maiusculas. Neste exemplo, ULSO
%      Parametro 3. A designação por extenso da sigla/acrónimo.
%   
%    \gls{ulso}            mostra a sigla. Na primeira utilização mostra por extenso e a sigla
%    \glsentryshort{ulso}  Mostra apenas a sgila mesmo que seja a primeira utilização
%    \glsentrylong{ulso}   Mostra esempre por extenso
%
% INSTRUCOES DE USO:
%    No texto, quando queremos usar uma sigla devemos usar /gls{label}. Na primeira utilização, apareceça por extenso, nas subsequentes
%    aparece apenas a sigla.
%
%
% EXEMPLO:
%    \newacronym{ulso}{ULSO}{Unidade Local de Saúde do Oeste}
%    \newacronym{DGS}{DGS}{Direção-Geral da Saúde}
%
%    A \gls{ulso} trabalha em articulação estreita com a \gls{DGS}. Existem Médicos de Saúde Pública a 
%    trabalhar tanto na \gls{ulso} na \gls{dgs}.
%
% output:
%    A Unidade Local de Saúde do Oeste (ULSO) trabalha em articulação estreita com a Direção-Geral da Saúde (DGS). 
%    Existem Médicos de Saúde Pública a trabalhar tanto na ULSO como na DGS.
%% ----------------------------------------------------------------------------
\newacronym{ulso}{ULSO}{Unidade Local de Saúde do Oeste}
\newacronym{DGS}{DGS}{Direção-Geral da Saúde}
\newacronym{sinave}{SINAVE}{Sistema Nacional de Vigilância Epidemiológica}
\newacronym{FOO}{FOO}{Foobar}
\newacronym{bar}{BAR}{Barista}
\newacronym{HTA}{HTA}{Hipertensão Arterial}


%% COLORS %%
% Define the green color for PROCEDIMENTO
\definecolor{procegreen}{HTML}{12654E}
\definecolor{tableheader}{HTML}{B6D7D2}

% Define a command to convert text to uppercase
\newcommand{\toUpper}[1]{\expandafter\uppercase\expandafter{#1}}

% Custom section formatting with green color
\usepackage{titlesec}
%\titleformat{\chapter}{\color{procegreen}\large\bfseries\uppercase}{\thechapter.}{1em}{}
\titleformat{\section}{\color{procegreen}\large\bfseries\uppercase}{\thesection}{1em}{}
\titleformat{\subsection}{\color{procegreen}\normalsize\bfseries\uppercase}{\thesubsection}{1em}{}
\titleformat{\subsubsection}{\color{procegreen}\normalsize\bfseries\uppercase}{\thesubsubsection}{1em}{}

% Contador para os anexos
\newcounter{anexocounter}
\renewcommand{\theanexocounter}{\Roman{anexocounter}}
\newcommand{\refanexo}[1]{\hyperref[#1]{Anexo~\ref*{#1}}}

\newcommand{\anexo}[1]{%
  \clearpage%
  \thispagestyle{fancy}%
  \refstepcounter{anexocounter}%
  \vspace{2em}%
  {\color{procegreen}\large\bfseries\uppercase{Anexo \theanexocounter. #1}\par}%
  \addcontentsline{toc}{subsection}{Anexo \theanexocounter. #1}%
  \vspace{1em}%
}


% Set global list styles for enumerate
\setlist{nosep}
\setlist[enumerate,1]{label=\arabic*., topsep=8pt,itemsep=0pt,partopsep=0pt, parsep=4pt}
\setlist[enumerate,2]{label=\alph*), topsep=0pt,itemsep=0pt,partopsep=0pt, parsep=4pt}
\setlist[enumerate,3]{label=\roman*., topsep=0pt,itemsep=0pt,partopsep=0pt, parsep=4pt}

\setlength{\parindent}{0pt}

% Remove chapter/section numbering
\setcounter{secnumdepth}{0}

% Format the TOC entries without numbers
\usepackage{tocloft}
\renewcommand{\cftchappresnum}{}
\renewcommand{\cftsecpresnum}{}
\renewcommand{\cftsubsecpresnum}{}
\renewcommand{\cftsubsubsecpresnum}{}


% Setup the header with the logos
\usepackage{fancyhdr}
\pagestyle{fancy}
\fancyhf{}
\renewcommand{\headrulewidth}{0pt}

% Create the header with logo images
\fancyhead[L]{%
  \begin{minipage}{0.75\textwidth}
    \raisebox{-0.15cm}{\includegraphics[height=1cm]{media/republica_portuguesa.png}}%
    \hspace{0.2cm}%
    \raisebox{0pt}{\includegraphics[height=1cm]{media/SNS_logocurto_cor.png}}%
    \hspace{0.2cm}%
    \raisebox{0.11cm}{\includegraphics[height=0.83cm]{media/uls_oeste_logo.png}}%
  \end{minipage}%
}
\fancyhead[R]{%
  \begin{minipage}{0.4\textwidth}
    \raggedleft PROCEDIMENTO \procedimentoNumero
  \end{minipage}%
}
\fancyfoot[C]{\thepage}


% Ficha Técnica page wrapped in a custom environment
\newenvironment{fichatecnica}{%
  \clearpage
  \thispagestyle{fancy}
  \vspace*{-0.5cm}
  {\Large\bfseries\color{procegreen}FICHA TÉCNICA\par}
  \vspace{1cm}
}{%
  \clearpage
}

% capa dos anexos
\newenvironment{capaanexos}{
  \clearpage
  \pagestyle{empty}
  \begin{tikzpicture}[remember picture,overlay]
}{
  \end{tikzpicture}
  \restoregeometry
  \clearpage
  \pagestyle{fancy}
  \thispagestyle{fancy}
}


%% para adicionar e remover entradas no TOC
\let\oldaddcontentsline\addcontentsline
\newcommand{\stoptocentries}{\renewcommand{\addcontentsline}[3]{}}
\newcommand{\starttocentries}{\let\addcontentsline\oldaddcontentsline}

% Alterar "Tabela" para "Quadro" em todas as legendas de tabelas
\renewcommand{\tablename}{Quadro}

% WATERMARK

\ifdefined\estadodraft%
    \usepackage{draftwatermark}
    \SetWatermarkText{DRAFT}
    \SetWatermarkScale{2}
\fi


\begin{document}

% numeracao para ficha tecnica, toc, etc...
\pagenumbering{roman} % Start with roman numerals

%%%%%%%%%%%%%%%%%%%%%%%%%%%%%%%%%%%%%%%%%%%%%%%%%%%%%%%%%%%%%%%%%%%%%%%%%%%%%%%%%%%%%%%
%% CAPA
%%%%%%%%%%%%%%%%%%%%%%%%%%%%%%%%%%%%%%%%%%%%%%%%%%%%%%%%%%%%%%%%%%%%%%%%%%%%%%%%%%%%%%%

\begin{titlepage}
\newgeometry{top=0cm, bottom=0cm, left=0cm, right=0cm, noheadfoot}
\begin{tikzpicture}[remember picture,overlay]
  % Place background image covering the entire page
  \node[anchor=center] at (current page.center) {
    \includegraphics[width=\paperwidth,height=\paperheight]{media/fundo_capa.png}
  };
  
  % Add the title text
  \node[anchor=north west, text width=0.75\paperwidth, align=left] 
    at ([xshift=3cm, yshift=-8cm]current page.north west) {
    \fontsize{16}{20}\selectfont\bfseries\color{procegreen}
    PROCEDIMENTO \procedimentoNumero \\[0.4cm]
    \fontsize{14}{18}\selectfont\bfseries\color{black}
    \procedimentoTitulo \\[0.3cm]
    \fontsize{12}{16}\selectfont\mdseries
    Versão \procedimentoVersao\\[0.5cm]
    \fontsize{12}{16}\selectfont\mdseries
    \procedimentoData\\
  };

\end{tikzpicture}
\restoregeometry
\end{titlepage}


%%%%%%%%%%%%%%%%%%%%%%%%%%%%%%%%%%%%%%%%%%%%%%%%%%%%%%%%%%%%%%%%%%%%%%%%%%%%%%%%%%%%%%%%%

% Fix for header display on all pages
\pagestyle{fancy}  % Set fancy style as default for all pages

%%%%%%%%%%%%%%%%%%%%%%%%%%%%%%%%%%%%%%%%%%%%%%%%%%%%%%%%%%%%%%%%%%%%%%%%%%%%%%%%%%%%%%%%%
%% FICHA TECNICA
%%%%%%%%%%%%%%%%%%%%%%%%%%%%%%%%%%%%%%%%%%%%%%%%%%%%%%%%%%%%%%%%%%%%%%%%%%%%%%%%%%%%%%%%%

% Usage in the document
\begin{fichatecnica}
{\large\bfseries TÍTULO\par}
\procedimentoTitulo
\vspace{0.5cm}

{\large\bfseries NÚMERO\par}
\procedimentoNumero
\vspace{0.5cm}

{\large\bfseries VERSÃO\par}
Versão \procedimentoVersao
\vspace{0.5cm}

{\large\bfseries DESTINATÁRIOS\par}
\procedimentoDestinatarios
\vspace{0.5cm}

{\large\bfseries PALAVRAS-CHAVE\par}
\procedimentoPalavrasChave
\vspace{0.5cm}

{\large\bfseries AUTORES\par}
\listarAutores
\vspace{0.5cm}

{\large\bfseries CONTACTOS\par}
Departamento de Saúde Pública e das Populações\\
Rua Diário de Notícias\\
2500-176 Caldas da Rainha\\
Tel.: 262 248 840 / 261 336 370\\
E-mail: dsppoeste@ulso.min-saude.pt\\
\vspace{0.5cm}

{\large\bfseries CONTROLO DE PUBLICAÇÕES\par}
\vspace{0.3cm}
\begin{center}
\tabelaControloVersao
\end{center}
\end{fichatecnica}

%%%%%%%%%%%%%%%%%%%%%%%%%%%%%%%%%%%%%%%%%%%%%%%%%%%%%%%%%%%%%%%%%%%%%%%%%%%%%%%%%%%%%%%%%


%%%%%%%%%%%%%%%%%%%%%%%%%%%%%%%%%%%%%%%%%%%%%%%%%%%%%%%%%%%%%%%%%%%%%%%%%%%%%%%%%%%%%%%%%
%% ÍNDICE (TOC)
%%%%%%%%%%%%%%%%%%%%%%%%%%%%%%%%%%%%%%%%%%%%%%%%%%%%%%%%%%%%%%%%%%%%%%%%%%%%%%%%%%%%%%%%%
\clearpage
% Customize TOC title
\renewcommand{\contentsname}{\color{procegreen}\Large\bfseries\uppercase{ÍNDICE}}
\vspace*{-2cm}
\tableofcontents
\thispagestyle{fancy}  % Force fancy style after tableofcontents

%%%%%%%%%%%%%%%%%%%%%%%%%%%%%%%%%%%%%%%%%%%%%%%%%%%%%%%%%%%%%%%%%%%%%%%%%%%%%%%%%%%%%%%%%
%% ÍNDICE DE QUADROS E FIGURAS
%%%%%%%%%%%%%%%%%%%%%%%%%%%%%%%%%%%%%%%%%%%%%%%%%%%%%%%%%%%%%%%%%%%%%%%%%%%%%%%%%%%%%%%%%
\ifdefined\usarindicequadros
    \renewcommand{\listtablename}{\color{procegreen}\Large\bfseries\uppercase{ÍNDICE DE QUADROS}}
    
    % List of tables (quadros)
    \clearpage
    \vspace*{-2cm}
    \listoftables
    \thispagestyle{fancy}  % Force fancy style after listoftables
\fi

\ifdefined\usarindicefiguras
    \renewcommand{\listfigurename}{\color{procegreen}\Large\bfseries\uppercase{ÍNDICE DE FIGURAS}}
    \vspace{1.5cm}
    % List of figures
    \listoffigures
    \thispagestyle{fancy}  % Force fancy style after listoffigures
\fi

%%%%%%%%%%%%%%%%%%%%%%%%%%%%%%%%%%%%%%%%%%%%%%%%%%%%%%%%%%%%%%%%%%%%%%%%%%%%%%%%%%%%%%%%%
%% SIGLAS E ACRÓNIMOS
%%%%%%%%%%%%%%%%%%%%%%%%%%%%%%%%%%%%%%%%%%%%%%%%%%%%%%%%%%%%%%%%%%%%%%%%%%%%%%%%%%%%%%%%%
\ifdefined\usarsiglas
    \clearpage
    \thispagestyle{fancy}
    \vspace*{-0.5cm}
    {\color{procegreen}\Large\bfseries\uppercase{SIGLAS E ACRÓNIMOS}\par}
    \vspace{0.5cm}
    
    \begingroup
    \renewcommand*{\glossarysection}[2][]{} % impede qualquer título ou secção
    \printglossary[type=\acronymtype]
    \endgroup
\fi

\thispagestyle{fancy} % repõe caso o glossário tenha alterado o estilo
\clearpage

%% RESET PAGE NUMBERING TO ARABIC
\pagenumbering{arabic} % Switch to arabic numerals
\setcounter{page}{1} % Reset page counter to 1
\clearpage

%%%%%%%%%%%%%%%%%%%%%%%%%%%%%%%%%%%%%%%%%%%%%%%%%%%%%%%%%%%%%%%%%%%%%%%%%%%%%%%%%%%%%%%%%%%%
%% CONTEUDO
%%%%%%%%%%%%%%%%%%%%%%%%%%%%%%%%%%%%%%%%%%%%%%%%%%%%%%%%%%%%%%%%%%%%%%%%%%%%%%%%%%%%%%%%%%%%

\begin{onehalfspace}
    \setlist{topsep=1em, itemsep=1em}
    \let\oldsection\section
    \renewcommand{\section}{\phantomsection\oldsection}
    \let\oldsubsection\subsection
    \renewcommand{\subsection}{\phantomsection\oldsubsection}
    \section{Enquadramento}\label{sec:enquadramento}

A atividade do \gls{dspp} da \gls{ulso} requer uma abordagem sistematizada e documentada para a definição e implementação de procedimentos técnico-operacionais. Estes procedimentos visam assegurar coerência, qualidade e rastreabilidade nas intervenções em saúde pública, bem como promover a equidade e eficiência na atuação dos profissionais do \gls{dspp}.

O presente documento estabelece as normas e orientações para a proposição, elaboração, aprovação, alteração e revogação de procedimentos internos do \gls{dspp}. Este “procedimento de procedimentos” visa garantir que todos os documentos produzidos sigam uma metodologia uniforme e criteriosamente validada, promovendo a melhoria contínua dos processos organizacionais e técnicos no âmbito da saúde pública.

Este procedimento é aplicável a todas as tipologias de procedimentos operacionais internos, incluindo os de carácter clínico, técnico, administrativo ou transversal, independentemente da área temática ou programa de saúde a que digam respeito.


\section{Objetivo}\label{sec:objetivo}

O presente procedimento tem como objetivo normatizar o processo de criação, revisão, alteração e revogação de procedimentos internos no âmbito do \gls{dspp}


\section{Âmbito}\label{sec:ambito}

Entende-se por procedimento interno qualquer documento normativo, aprovado e homologado, que estabeleça de forma sistemática as etapas, responsabilidades, critérios técnicos e operacionais a observar na execução de uma determinada atividade no âmbito do \gls{dspp}.

Este procedimento aplica-se a todos os procedimentos internos desenvolvidos no seio do \gls{dspp}, independentemente da sua natureza (clínica, técnica, administrativa ou transversal) ou da área temática a que respeitem. Abrange igualmente os procedimentos produzidos no contexto dos programas de saúde sob coordenação do \gls{dspp}.

Estão incluídos no âmbito deste procedimento:

\begin{itemize}
    \item a criação de novos procedimentos por proposta de profissionais ou órgãos do \gls{dspp};
    \item a alteração ou atualização de procedimentos existentes;
    \item a revogação formal de procedimentos desatualizados, obsoletos ou redundantes;
    \item a definição da estrutura e conteúdo obrigatório de cada procedimento;
    \item os critérios de validação técnica e homologação hierárquica.
\end{itemize}


\section{Destinatários}\label{sec:destinatarios}

O presente procedimento destina-se a todos os profissionais do \gls{dspp}, incluindo:

\begin{itemize}
    \item Direção do Departamento;
    \item membros do Conselho Técnico;
    \item coordenadores e elementos das equipas de coordenação dos programas de saúde sob responsabilidade do \gls{dspp};
    \item médicos, enfermeiros, técnicos superiores de saúde, assistentes técnicos e outros profissionais integrados nos serviços e unidades funcionais do departamento;
    \item colaboradores externos que, mediante delegação ou participação formal, estejam envolvidos na elaboração, revisão ou implementação de procedimentos.
\end{itemize}

Este procedimento é de cumprimento obrigatório por todos os profissionais do \gls{dspp} envolvidos na elaboração, revisão ou implementação de procedimentos internos, bem como pelas equipas de coordenação dos programas operacionais da \gls{ulso} que atuem sob responsabilidade direta ou indireta do departamento.


\section{Definição}\label{sec:definicao}

    \subsection{Definição Funcional}\label{subsec:definicao-funcional}

    O presente procedimento define um instrumento que:
    \begin{itemize}
        \item define as etapas necessárias à criação, revisão, alteração e revogação de procedimentos internos;
        \item estrutura os fluxos de validação técnica e decisão hierárquica entre o Conselho Técnico e a Direção do \gls{dspp};
        \item assegura a uniformidade e consistência dos documentos produzidos, promovendo a sua legibilidade e aplicabilidade prática;
        \item garante a rastreabilidade das decisões e intervenções associadas a cada procedimento.
    \end{itemize}
    
    \subsection{Limites de Entrada no Processo}\label{subsec:limites-de-entrada-no-processo}

    Este procedimento tem início com a submissão formal da proposta de criação, alteração ou revogação de um procedimento ao Conselho Técnico, por correio eletrónico. Esta proposta deve identificar de forma clara a necessidade sentida, o objetivo do procedimento e os seus limites previstos.

    \subsection{Limites de Saída do Processo}\label{subsec:limites-de-saida-do-processo}

    O procedimento termina com a homologação pelo Diretor do Departamento e consequente publicação do procedimento final na pasta partilhada oficial do \gls{dspp}, acompanhada de notificação por email aos profissionais do departamento.

    \subsection{Limites Marginais}\label{subsec:limites-marginais}

    \begin{itemize}
        \item Registo da atividade relacionada com a elaboração do procedimentos (e dos procedimentos em si)
        \item Avaliação e Monitorização dos procedimentos
    \end{itemize}


\section{Responsabilidades}\label{sec:responsabilidades}

\begin{itemize}
    \item \textbf{Profissionais do \gls{dspp}}: Podem propor a criação, alteração ou revogação de procedimentos, devendo submeter proposta formal ao Conselho Técnico, com identificação clara da necessidade, objetivo e âmbito do procedimento.
    \item \textbf{Equipas de Coordenação dos Programas}: Podem propor e colaborar na elaboração de procedimentos no âmbito do respetivo programa. Assumem a redação técnica sempre que designadas para esse efeito pelo Conselho Técnico.
    \item \textbf{Conselho Técnico}: Recebe, analisa e delibera sobre as propostas de procedimento. Designa a equipa responsável pela sua elaboração, define prazos e acompanha o desenvolvimento do documento. Revê tecnicamente o procedimento e propõe alterações, submetendo a versão final ao Diretor do \gls{dspp} para homologação. Em caso de revogação, emite parecer e assegura a remoção do procedimento do repositório oficial.
    \item \textbf{Diretor do \gls{dspp}}: Homologa os procedimentos após apreciação da versão final. Pode solicitar alterações e devolvê-las ao Conselho Técnico. Tem competência para aprovar a revogação de procedimentos e determinar a sua entrada em vigor.
\end{itemize}


\section{Descrição do procedimento}\label{sec:descricao-do-procedimento}

\subsection{Criação de um novo procedimento}\label{sec:criacao-do-procedimento}

\begin{enumerate}
  \item Qualquer profissional, equipa de coordenação de programa, o \textbf{Conselho Técnico} ou o \textbf{Diretor do \gls{dspp}} pode propor a criação de um novo procedimento.
  \item A proposta deve ser enviada por e-mail ao \textbf{Conselho Técnico}, com a identificação da necessidade sentida, o objetivo do procedimento e os seus limites.
  \item Após recepção da proposta, o \textbf{Conselho Técnico} avalia a proposta e decide sobre a sua pertinência;
  \item Em caso de indeferimento, o \textbf{Conselho Técnico} comunica a decisão ao propositor, dando conhecimento ao \textbf{Diretor do \gls{dspp}};
  \item Em caso de deferimento, o \textbf{Conselho Técnico}:
  \begin{enumerate}
    \item Designa uma \textbf{Equipa Redatora}, responsável pela elaboração do procedimento;
    \item Designa um \textbf{Revisor}, responsável pela revisão do procedimento, que pode ser o próprio conselho técnico ou um elemento externo a este;
    \item Define um prazo para entrega do primeiro draft do documento e um prazo para finalização dos trabalhos;
    \item Atribui um título ao Procedimento;
    \item Atribui um número ao Procedimento, no formato PXXX.AAAA, em que XXX é um número sequencial composto por 3 digitos e AAAA correspondem aos 4 dígitos do ano corrente;
    \item Regista o procedimento na \textit{tabela dos procedimentos}, disponível na \href{https://snspt-my.sharepoint.com/:l:/r/personal/93839_ulso_min-saude_pt/Lists/Lista%20de%20Procedimentos?e=zkbTNC}{Lista de Procedimentos}, preenchendo as colunas de forma apropriada;
    \item Comunica, via email, a decisão ao propositor e à equipa redatora, dando conhecimento ao \textbf{Diretor do \gls{dspp}};
  \end{enumerate}
  \item A \textbf{Equipa Redatora} elabora o procedimento, seguindo o modelo oficial e cujas instruções se encontram no \refanexo{anexo-estrutura-modelo-procedimento}
  \item O documento (sem formatação final) é enviado por email ao \textbf{Conselho Técnico}.
  \item O \textbf{Conselho Técnico} analisa e propõe alterações, se aplicável. A comunicação deve seguir sempre na mesma cadeia de emails.
  \item Após validação, o \textbf{Conselho Técnico} formata o documento e submete-o ao \textbf{Diretor do \gls{dspp}} para homologação.
  \item O Diretor pode sugerir alterações e devolvê-las ao \textbf{Conselho Técnico}.
  \item O processo de revisão pode ser repetido até aprovação final.
  \item Após homologação do documento pelo \textbf{Diretor do \gls{dspp}}, o \textbf{Conselho Técnico}:
  \begin{enumerate}
    \item Atualiza o ficheiro "metadata.tex" com as seguintes informações: Versão 1.0.0; Data do procedimento com a data em que o documento foi homologado pelo Diretor do \gls{dspp}; Tabela de controlo de versões com o revisor, quem homologou o documento e a data de homologação.
    \item publica o procedimento na pasta partilhada do \gls{dspp}, na diretoria "/Procedimentos e Legislação";
    \item envia, via email, notificação da existência de um novo procedimento a todos os profissionais.
  \end{enumerate}
  
   e comunica-o, via email, a todos os profissionais.
\end{enumerate}

\subsection{Alteração de um procedimento existente}

\begin{enumerate}
  \item Qualquer profissional do \gls{dspp} pode propor alterações a um procedimento existente.
  \item A proposta deve ser enviada ao \textbf{Conselho Técnico}, por e-mail, com:
  \begin{enumerate}
    \item Justificação fundamentada para a alteração (racional da proposta);
    \item Documento original em anexo, com as alterações devidamente assinaladas (em comentário ou controlo de alterações);
  \end{enumerate}
  \item Após receção da proposta, o \textbf{Conselho Técnico} avalia a pertinência da mesma.
  \item Em caso de indeferimento, comunica a decisão ao propositor, dando conhecimento ao Diretor do \gls{dspp}.
  \item Em caso de deferimento, o \textbf{Conselho Técnico}:
  \begin{enumerate}
    \item Decide se, após validação, procede ele próprio à redação da nova versão ou se designa uma nova \textbf{Equipa Redatora};
    \item Define um prazo para entrega da versão revista;
    \item Comunica, via email, a decisão ao propositor e, se aplicável, à nova \textbf{Equipa Redatora}, dando conhecimento ao \textbf{Diretor do \gls{dspp}};
    \item Verte as alterações no "changelog.md" e atribui um novo número de versão, que segue as regras explanadas no \refanexo{anexo-regras-versionamento}
    \item Envia a nova versão para o \textbf{Diretor do \gls{dspp}} para homologação.
  \end{enumerate}
  \item Após homologação do documento pelo \textbf{Diretor do \gls{dspp}}, o \textbf{Conselho Técnico} publica a nova versão na pasta partilhada do \gls{dspp} e comunica-o, via email, a todos os profissionais.
\end{enumerate}


\section{Situações de Excepção}\label{sec:situacoes-de-excepcao}


\section{Monitorização e Avaliação}\label{sec:monitorizacao-e-avaliacao}


\end{onehalfspace}

%%%%%%%%%%%%%%%%%%%%%%%%%%%%%%%%%%%%%%%%%%%%%%%%%%%%%%%%%%%%%%%%%%%%%%%%%%%%%%%%%%%%%%%%%%%%%

\section{Documentos de Referência}\label{sec:referencias}
\printbibliography[heading=none]

\newpage

% changelog
\clearpage
\section{Registo de Alterações}\label{sec:registo-de-alteracoes}
\nopagebreak
\stoptocentries% Stop adding content to the ToC

% Define section and subsection titles without numbering
\titleformat{\section}[block]{\normalfont\Large\bfseries}{}{0em}{}
\titleformat{\subsection}[block]{\normalfont\large\bfseries}{}{0em}{}

\markdownInput{changelog.md}

% Restore the original section and subsection titles
\titleformat{\section}[block]{\normalfont\Large\bfseries}{\thesection}{1em}{}
\titleformat{\subsection}[block]{\normalfont\large\bfseries}{\thesubsection}{1em}{}

\starttocentries% Resume adding content to the ToC

\ifdefined\usaranexos
    %%%%%%%%%%%%%%%%%%%%%%%%%%%%%%%%%%%%%%%%%%%%%%%%%%%%%%%%%%%%%%%%%%%%%%%%%%%%%%%%%%%%%%%
    %% CAPA ANEXOS
    %%%%%%%%%%%%%%%%%%%%%%%%%%%%%%%%%%%%%%%%%%%%%%%%%%%%%%%%%%%%%%%%%%%%%%%%%%%%%%%%%%%%%%%
    \newpage
    \begin{capaanexos}
      % Imagem de fundo
      \node[anchor=center] at (current page.center) {
        \includegraphics[width=\paperwidth,height=\paperheight]{media/fundo2.png}
      };
    
      % Título
      \node[anchor=north west, text width=0.75\paperwidth, align=left] 
        at ([xshift=3cm, yshift=-20cm]current page.north west) {
        \fontsize{30}{40}\selectfont\bfseries\color{procegreen} ANEXOS\\
      };
    
      % Entrada no índice
      \phantomsection
      \addcontentsline{toc}{section}{Anexos}
    \end{capaanexos}
    
    \anexo{Quadro de Indicadores}

Aqui entra o conteúdo do Anexo I.

\pagebreak

\anexo{Tabela Comparativa de Normas}

E aqui o conteúdo do Anexo II.
\fi



%%%%%%%%%%%%%%%%%%%%%%%%%%%%%%%%%%%%%%%%%%%%%%%%%%%%%%%%%%%%%%%%%%%%%%%%%%%%%%%%%%%%%%%
%% CONTRA CAPA
%%%%%%%%%%%%%%%%%%%%%%%%%%%%%%%%%%%%%%%%%%%%%%%%%%%%%%%%%%%%%%%%%%%%%%%%%%%%%%%%%%%%%%%

\begin{capaanexos}
  % Imagem de fundo
  \node[anchor=center] at (current page.center) {
    \includegraphics[width=\paperwidth,height=\paperheight]{media/contracapa.jpg}
  };
\end{capaanexos}


\end{document}
