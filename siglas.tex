%% ----------------------------------------------------------------------------
% COMANDOS:
%
%    \newacronym{ulso}{ULSO}{Unidade Local de Saúde do Oeste}
%      Parametro 1. O label para referir no texto com o comando gls{}. Neste exemplo, gls{ulso}
%      Parametro 2. A sigla/acrónimo. Geralmente em maiusculas. Neste exemplo, ULSO
%      Parametro 3. A designação por extenso da sigla/acrónimo.
%   
%    \gls{ulso}            mostra a sigla. Na primeira utilização mostra por extenso e a sigla
%    \glsentryshort{ulso}  Mostra apenas a sgila mesmo que seja a primeira utilização
%    \glsentrylong{ulso}   Mostra esempre por extenso
%
% INSTRUCOES DE USO:
%    No texto, quando queremos usar uma sigla devemos usar /gls{label}. Na primeira utilização, apareceça por extenso, nas subsequentes
%    aparece apenas a sigla.
%
%
% EXEMPLO:
%    \newacronym{ulso}{ULSO}{Unidade Local de Saúde do Oeste}
%    \newacronym{DGS}{DGS}{Direção-Geral da Saúde}
%
%    A \gls{ulso} trabalha em articulação estreita com a \gls{DGS}. Existem Médicos de Saúde Pública a 
%    trabalhar tanto na \gls{ulso} na \gls{dgs}.
%
% output:
%    A Unidade Local de Saúde do Oeste (ULSO) trabalha em articulação estreita com a Direção-Geral da Saúde (DGS). 
%    Existem Médicos de Saúde Pública a trabalhar tanto na ULSO como na DGS.
%% ----------------------------------------------------------------------------
\newacronym{ulso}{ULSO}{Unidade Local de Saúde do Oeste}
\newacronym{dspp}{DSPP}{Departamento de Saúde Pública e das Populações}
\newacronym{conselhotecnico}{Conselho Técnico}{Conselho Técnico do Departamento de Saúde Pública e das Populações}
